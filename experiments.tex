\documentclass{article}
\usepackage[utf8]{inputenc}
\usepackage{url}
\usepackage{hyperref}

\title{Numerical experiments - market ecology}
\author{Aymeric Vi\'{e}}
\date{\today}

\begin{document}

\maketitle

\tableofcontents

\clearpage

\section{Preambule: baseline configuration}

Necessary condition: running GA with complete market structure
Objective: be as close as possible to Maarten's paper configuration, notably for market environment parameters.
Add Agent leverage, aggressiveness of response to be closer to Maarten
\section{Preambule 2: observables}

Necessary condition: running GA with complete market structure
To implement:
\begin{itemize}
    \item Observe market dynamics over time (easy): price, returns, assets, volatility, dividends, profits
    \item Observe ecology dynamics over time (hard): proportions? heatmap?
    Perhaps H (for max time horizon) plots with clear color codes (like jet) over time, and hoepfully we see something clear happen.
    \item Develop a GUI or something nice/updating iteself over time to observe that?
\end{itemize}

\section{Trend follower market ecology}

Necessary condition: a running GA with a single asset, market clearing, and adequate starting conditions, and the baseline configuration.

\subsection{Influence of learning vs non-learning}
The non-lerning case (replacement dynamics only) can be illustrated with:
\begin{itemize}
    \item No selection (selection rate equal to 0)
    \item No crossover (crossover rate equal to 0)
    \item No mutation (mutation rate equal to 0)
\end{itemize}
The objective is to assess the behavior of the model in this no-learning setup, establish a benchmark.

Then, we can introduce learning at different degrees, and channels. Learning requires activation of selection.

\begin{enumerate}
    \item Mutation alone, with different values
    \item Crossover alone, with different values
    \item Mutation and crossover
\end{enumerate}

How does learning affect market dynamics? Ecology dynamics?

There are many many things we can also study:
\begin{itemize}
    \item Influence of strategy space with H
    \item Time horizon of EMA fitness profit
    \item Market parameters: dividend growth rate, volatility, interest rate
    \item Alternative values for agent leverage, aggressiveness of response 
\end{itemize}

\end{document}